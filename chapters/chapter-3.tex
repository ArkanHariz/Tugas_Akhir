\newpage
\chapter{METODE PENELITIAN} \label{Bab III}

\section{Alur Penelitian} \label{III.Alur}
Penelitian ini berfokus pada augmentasi data kelas minoritas untuk 
membandingkan hasil atau performa arsitektur ResNet-18 dan ResNet-50. 
Alur penelitian dijelaskan pada diagram yang dapat dilihat pada Gambar \ref{fig:3.alur}.
Penelitian dimulai dari identifikasi masalah hingga selesai.

\begin{figure}[H] % Kalau menggunakan H, posisi gambar akan tepat dibawah teks
    \centering
    \includegraphics[width=1.0\textwidth]{figure/alur_penelitian.png}
    \caption{Diagram Alur Penelitian}
    \label{fig:3.alur}
\end{figure}

\section{Penjabaran Langkah Penelitian} \label{III.Jabar Alur}
Penjelasan dari setiap langkah pada alur penelitian akan dijelaskan 
secara rinci untuk memberikan detail yang jelas terhadap metode yang digunakan. 
Berikut penjelasan dari setiap langkah penelitian tersebut. \par

\subsection{Identifikasi Masalah} \label{III.Langkah 1}
Tahap awal yang dilakukan sebelum penelitian adalah identifikasi masalah. 
Identifikasi dilakukan dengan cara menelaah karakteristik dataset DermaMNIST
yang digunakan dalam penelitian klasifikasi citra penyakit kulit. Pada 
penelitian klasifikasi citra penyakit kulit menggunakan dataset DermaMNIST 
yang dilakukan oleh Nerma Kadric dkk pada tahun 2025, terdapat 
ketidakseimbangan kelas yang signifikan di mana berdampak terhadap performa 
model \cite{kadric2025improvingdiagnosticaccuracypigmented}. Berdasarkan hal 
tersebut, dilakukan eksplorasi terhadap dataset dan ditemukannya kelas yang
jumlah data nya sedikit (kelas minoritas) dibandingkan kelas lainnya yang 
memiliki jumlah data yang banyak (kelas mayoritas). Kondisi 
ketidakseimbangan kelas ini menjadi dasar dilakukannya penelitian untuk 
mengevaluasi efektivitas teknik augmentasi pada kelas minoritas dalam 
membandingkan performa model ResNet-18 dan ResNet-50 pada klasifikasi citra 
penyakit kulit menggunakan dataset DermaMNIST. \par

\subsection{Studi Literatur} \label{III.Langkah 2}
Tahap studi literatur dilakukan dengan mengumpulkan dan mengkaji berbagai 
sumber akademis terpercaya, meliputi jurnal ilmiah dan publikasi terkini 
yang relevan dengan penelitian. Referensi yang dikumpulkan berdasarkan pada
topik klasifikasi citra penyakit kulit menggunakan deep learning, imbalanced 
data pada multi-kelas, penanganan imbalanced data melalui teknik augmentasi, 
dan model ResNet-18 dan ResNet-50. Tahap ini bertujuan untuk membangun 
landasan teoritis yang kuat dan berkaitan dengan penelitian yang dilakukan. \par

\subsection{Akuisisi Dataset} \label{III.Langkah 3}
Proses pengumpulan data dalam riset ini menggunakan dataset DermaMNIST yang 
diambil dari platform MedMNIST, yaitu open-source yang menyediakan koleksi 
dataset citra medis.  Dataset DermaMNIST berisi gambar-gambar penyakit kulit 
berwarna dengan dimensi 28x28 piksel dan 224x224 piksel, yang dikelompokkan 
menjadi 7 kelas berbeda sesuai jenis penyakit kulitnya. Dataset ini terdiri 
dari total 10.015 gambar yang terbagi menjadi data latih, data validasi, dan 
data uji. Ketujuh kelas penyakit kulit tersebut meliputi actinic keratosis 
and intraepithelial carcinoma, basal cell carcinoma, benign keratosis-like 
lesions, dermatofibroma, melanoma, melanocytic nevi, dan vascular lesions. \par

\subsection{Preprocessing Data} \label{III.Langkah 4}
Sebelum pelatihan model melakukan dataset yang sudah dikumpulkan, data akan 
diolah terlebih dahulu. Tahapan tersebut disebut dengan \textit{Preprocessing 
Data}. Tahap \textit{preprocessing data} akan melakukan \textit{resizing} dan 
normalisasi data. Pada tahap \textit{resizing}, data di-\textit{resize} dari 
ukuran aslinya yaitu 28x28 menjadi 224x224 agar sesuai dengan dimensi input 
\textit{default} pada arsitektur ResNet, baik ResNet-18 maupun ResNet-50. 
Proses normalisasi dilakukan untuk mentransformasi rentang nilai piksel dari 
[0, 255] menjadi [0, 1] dengan tujuan menstabilkan konvergensi selama 
pelatihan model dan meningkatkan efisiensi komputasi. Dataset DermaMNIST yang 
digunakan dalam penelitian ini terdiri dari 10.015 citra. Pembagian data 
mengikuti pembagian yang sudah disediakan oleh dataset, yaitu 7.007 citra 
untuk data \textit{training}, 1.003 citra untuk data \textit{validation}, 
dan 2.005 citra untuk data \textit{test}. \par

\subsection{Augmentasi Kelas Minoritas} \label{III.Langkah 5}
Augmentasi data merupakan teknik untuk meningkatkan jumlah data pelatihan 
dengan menghasilkan variasi baru dari data yang sudah tersedia melalui 
berbagai modifikasi seperti rotasi, \textit{flipping}, dan \textit{cropping}. 
Teknik ini bertujuan untuk memperbesar ukuran dan meningkatkan keragaman 
dataset sehingga dapat mengurangi \textit{overfitting} dan meningkatkan 
performa model \cite{MAHARANA202291}. Pada penelitian ini, augmentasi hanya 
diterapkan pada data di kelas minoritas. Metode yang digunakan untuk 
augmentasi meliputi rotasi, \textit{flip}, dan \textit{random erasing}. 
Gambar akan dirotasi secara acak dengan rentang sudut 0° hingga 45° derajat. 
Pada metode \textit{flip}, gambar akan di-\textit{flip} secara vertikal dan 
horizontal. Pada metode \textit{random erasing}, gambar akan diberikan area 
persegi panjang dengan rasio aspek acak, kemudian nilai piksel asli diganti 
dengan nilai piksel acak (\textit{noise}) atau rata-rata dataset 
(\textit{mean}) \cite{tumewu2020klasifikasi}. \par

\subsection{Perancangan Model ResNet-18 dan ResNet-50} \label{III.Langkah 6}
Tahapan ini merupakan penjelasan pada perancangan model ResNet-18 dan ResNet-50. 
Model ResNet menggunakan mekanisme \textit{residual learning} dengan 
\textit{skip connections} untuk mengatasi masalah \textit{vanishing gradient}, 
dimana ResNet-18 memiliki 18 lapisan dan ResNet-50 memiliki 50 lapisan. 
Perancangan diawali dengan pembagian dataset menjadi data \textit{training}, 
data \textit{validation}, dan data \textit{test}. Data \textit{training} 
akan dilakukan augmentasi data pada kelas minoritas. Setelah pembagian dan 
augmentasi, dataset digunakan sebagai input pada model ResNet-18 dan ResNet-50 
dan model melakukan \textit{training}. Hasil \textit{training} model akan 
disimpan dan dievaluasi untuk membandingkan performa antara kedua model. \par

\subsection{Evaluasi} \label{III.Langkah 7}
Tahapan ini adalah penjelasan terkait evaluasi yang akan digunakan pada model. 
Evaluasi terhadap model dilakukan menggunakan tiga cara, yakni \textit{Confusion 
Matrix}, metrik evaluasi, dan \textit{Matthews Correlation Coefficient} (MCC). 
\textit{Confusion Matrix} digunakan untuk mengevaluasi kinerja model guna 
memberikan rincian mengenai tingkat efektivitas model. Nilai yang dihasilkan 
dari \textit{Confusion Matrix} berupa nilai \textit{True Positive}, 
\textit{False Positive}, \textit{True Negative}, dan \textit{False Negative}. 
Nilai-nilai tersebut kemudian digunakan sebagai dasar perhitungan metrik 
evaluasi standar yang meliputi Akurasi, Presisi, \textit{Recall}, dan 
\textit{F1-Score}. Evaluasi ini juga menerapkan \textit{Matthews Correlation 
Coefficient} (MCC). MCC digunakan karena kemampuannya memberikan penilaian 
yang lebih pada data yang tidak seimbang dibandingkan akurasi biasa. \par

\section{Alat dan Bahan Tugas Akhir} \label{III.Alat dan Bahan}
Berisi alat-alat dan bahan-bahan yang digunakan dalam penelitian. \par

\subsection{Alat} \label{III.Alat}
Berikut merupakan alat-alat yang digunakan dalam pelaksanaan penelitian: \par
\begin{enumerate}[noitemsep]
	\item Laptop dengan spesifikasi windows 11, processor Intel(R) Core(TM) 
	i5-11400H 11th Gen @ 2.70GHz, memory 16 GB, dan graphics card 
	NVIDIA GeForce GTX 1650, SSD 512 GB
	\item Code editor Microsoft Visual Studio Code
	\item Python 3.10
	\item Libray python: PyTorch, NumPy, Matplotlib.
\end{enumerate}

\subsection{Bahan} \label{III.Bahan}
Penulis menggunakan dataset DermaMNIST yang merupakan bagian dari MedMNIST. 
Dataset terdiri dari citra penyakit kulit dengan 7 kelas berbeda, yaitu 
\textit{melanocytic nevi} sebanyak 6705 gambar, 
\textit{melanoma} sebanyak 1113 gambar, 
\textit{benign keratosis-like lesions} sebanyak 1099 gambar,
\textit{basal cell carcinoma} sebanyak 514 gambar,
\textit{actinic keratoses and intraepithelial carcinoma} sebanyak 327 gambar,
\textit{vascular lesions} sebanyak 142 gambar, dan 
\textit{dermatofibroma} sebanyak 115 gambar.
Dataset ini digunakan untuk pelatihan dan pengujian model klasifikasi. DermaMNIST
tersedia secara open access untuk digunakan dalam penelitian. \par

\section{Metode Pengembangan} \label{III.Metode}
Metode yang akan digunakan pada penelitian adalah metode \textit{Convolutional 
Neural Network} (CNN) dengan arsitektur ResNet-18 dan ResNet-50 untuk 
klasifikasi jenis penyakit kulit berdasarkan dataset DermaMNIST yang dimana 
membandingkan performa kedua arsitektur sebelum dan sesudah dilakukan 
augmentasi data yang dikhususkan untuk kelas minoritas. \par

\section{Ilustrasi Perhitungan Metode} \label{III.Ilustrasi}
Berikut adalah ilustrasi perhitungan metode yang digunakan dalam penelitian. \par

\subsection{Ilustrasi Perhitungan Konvolusi} \label{III.Konvolusi}
Perhitungan lapisan konvolusi dilakukan dengan cara mengalikan matriks filter 
(\textit{kernel}) dengan matriks input pada gambar. Misalkan terdapat matriks gambar 
sebesar 7x7 dan dan kernel 5x5. Selain itu, terdapat parameter lain yaitu 
\textit{padding} dengan nilai 0 dan \textit{stride} sebanyak 1 kali 
\textit{kernel} bergeser. Contoh matriks input pada gambar 7x7 dapat dilihat 
pada Gambar \ref{fig:3.konvolusi}.

\begin{figure}[H] % Kalau menggunakan H, posisi gambar akan tepat dibawah teks
    \centering
    \includegraphics[width=0.6\textwidth]{figure/lapisan_konvolusi.png}
    \caption{Matriks Input Gambar 7x7}
    \label{fig:3.konvolusi}
\end{figure}

Kemudian terdapat \textit{kernel} 5x5 yang akan digunakan untuk proses 
konvolusi. Contoh matriks \textit{kernel} dapat dilihat pada Gambar \ref{fig:3.kernel}.

\begin{figure}[H] % Kalau menggunakan H, posisi gambar akan tepat dibawah teks
    \centering
    \includegraphics[width=0.5\textwidth]{figure/kernel.png}
    \caption{Matriks Kernel 7x7}
    \label{fig:3.kernel}
\end{figure}

Selanjutnya dilakukan proses konvolusi dengan mengalikan matriks 
\textit{kernel} dengan matriks input gambar. Hasil perkalian antar kedua 
matriks akan dijumlahkan dan menghasilkan sebuah nilai pertama dari hasil 
proses konvolusi. Proses konvolusi dapat dilihat pada Gambar \ref{fig:3.contohkonv1}.

\begin{figure}[H] % Kalau menggunakan H, posisi gambar akan tepat dibawah teks
    \centering
    \includegraphics[width=1.0\textwidth]{figure/proses_konvolusi.png}
    \caption{Contoh Hasil Konvolusi}
    \label{fig:3.contohkonv1}
\end{figure}

Berikut adalah perhitungan dari proses konvolusi pertama: \\
\begin{align*} 
    &(4 \times 1) + (6 \times 1) + (3 \times 1) + (8 \times 1) + 
    (5 \times 1) + (9 \times 1) \\ &+ (1 \times (-2)) + (4 \times (-2)) + 
    (6 \times 2) + (3 \times 1) + (5 \times 1) + (7 \times (-2)) \\ 
    &+ (2 \times (-6)) + (9 \times (-2)) + (1 \times 1) + (3 \times 1) + 
    (8 \times 2) + (5 \times (-2)) \\ &+ (7 \times (-2)) + (2 \times 1) + 
    (6 \times 1) + (3 \times 1) + (8 \times 1) + (5 \times 1) \\ 
    &+ (7 \times 1) = 28 
\end{align*} \\

Proses konvolusi dilakukan secara berulang sesuai dengan parameter yang sudah 
ditentukan yaitu dengan pergeseran \textit{stride} sebanyak 1 kali dan tanpa 
\textit{padding}. Hasil akhir dari proses konvolusi ini adalah sebuah matriks 
output berukuran 3x3. Hasil akhir dari proses konvolusi dapat dilihat pada 
Gambar .

\begin{figure}[H] % Kalau menggunakan H, posisi gambar akan tepat dibawah teks
    \centering
    \includegraphics[width=1.0\textwidth]{figure/proses_konvolusi_2.png}
    \caption{Contoh Hasil Konvolusi}
    \label{fig:3.contohkonv2}
\end{figure}

\begin{figure}[H] % Kalau menggunakan H, posisi gambar akan tepat dibawah teks
    \centering
    \includegraphics[width=0.4\textwidth]{figure/hasil_akhir_konvolusi.png}
    \caption{Hasil Akhir Proses Konvolusi}
    \label{fig:3.endkonv}
\end{figure}

Berdasarkan gambar \ref{fig:3.endkonv} menunjukkan bahwa hasil akhir dari 
proses konvolusi menghasilkan matriks output berukuran 3x3 yang dimana matriks 
yang dihasilkan lebih kecil dibandingkan dengan matriks input awal yang 
berukuran 7x7. Ukuran matriks output pada proses konvolusi dapat dihitung 
menggunakan rumus berikut:

$$Output Size = \frac{I - F + 2P}{S} + 1$$
\\
\noindent Keterangan: \\
I = Ukuran Input \\ 
F = Ukuran Kernel \\ 
P = Padding \\ 
S = Stride

\subsection{Ilustrasi Perhitungan Fungsi Aktivasi ReLU} \label{III.RELU}
Fungsi aktivasi ReLU (Rectified Linear Unit) beroperasi dengan mengubah 
setiap nilai negatif dalam matriks menjadi 0, sedangkan nilai positif tidak 
mengalami perubahan. Fungsi ini diterapkan pada lapisan konvolusi dengan 
tujuan menambahkan non-linearitas ke dalam arsitektur model \cite{Yang_2023}. Berikut 
contoh perhitungan fungsi aktivasi ReLU pada output yang dihasilkan dari 
lapisan konvolusi sebelumnya.

\begin{figure}[H] % Kalau menggunakan H, posisi gambar akan tepat dibawah teks
    \centering
    \includegraphics[width=0.8\textwidth]{figure/fungsi_aktivasi.png}
    \caption{Visualisasi Fungsi Aktivasi ReLU}
    \label{fig:3.funcact}
\end{figure}

\subsection{Ilustrasi Perhitungan Pooling} \label{III.POOLING}
Proses pada lapisan Pooling dilakukan menggunakan teknik max pooling, yaitu 
dengan mengekstrak nilai tertinggi dari setiap wilayah pada matriks yang 
telah melalui fungsi aktivasi ReLU. Kernel berukuran 2x2 digunakan dalam 
lapisan Pooling ini dengan parameter stride bernilai 1. Contoh implementasi 
perhitungan pada lapisan Pooling disajikan pada ilustrasi berikut:

\begin{figure}[H] % Kalau menggunakan H, posisi gambar akan tepat dibawah teks
    \centering
    \includegraphics[width=1.0\textwidth]{figure/max_pooling.png}
    \caption{Visualisasi Perhitungan Pooling}
    \label{fig:3.pooling}
\end{figure} 

% \section{Rancangan Pengujian} \label{III.Rancang_Uji}
% Penjabaran terkait rancangan pengujian, pengujian perangkat keras, lunak, fungsional, dan non-fungsional. Berikan juga hipotesis hasil yang diharapkan dari penelitian. \par
